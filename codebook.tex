\documentclass[]{article}
\usepackage{lmodern}
\usepackage{amssymb,amsmath}
\usepackage{ifxetex,ifluatex}
\usepackage{fixltx2e} % provides \textsubscript
\ifnum 0\ifxetex 1\fi\ifluatex 1\fi=0 % if pdftex
  \usepackage[T1]{fontenc}
  \usepackage[utf8]{inputenc}
\else % if luatex or xelatex
  \ifxetex
    \usepackage{mathspec}
    \usepackage{xltxtra,xunicode}
  \else
    \usepackage{fontspec}
  \fi
  \defaultfontfeatures{Mapping=tex-text,Scale=MatchLowercase}
  \newcommand{\euro}{€}
\fi
% use upquote if available, for straight quotes in verbatim environments
\IfFileExists{upquote.sty}{\usepackage{upquote}}{}
% use microtype if available
\IfFileExists{microtype.sty}{%
\usepackage{microtype}
\UseMicrotypeSet[protrusion]{basicmath} % disable protrusion for tt fonts
}{}
\ifxetex
  \usepackage[setpagesize=false, % page size defined by xetex
              unicode=false, % unicode breaks when used with xetex
              xetex]{hyperref}
\else
  \usepackage[unicode=true]{hyperref}
\fi
\hypersetup{breaklinks=true,
            bookmarks=true,
            pdfauthor={},
            pdftitle={},
            colorlinks=true,
            citecolor=blue,
            urlcolor=blue,
            linkcolor=magenta,
            pdfborder={0 0 0}}
\urlstyle{same}  % don't use monospace font for urls
\setlength{\parindent}{0pt}
\setlength{\parskip}{6pt plus 2pt minus 1pt}
\setlength{\emergencystretch}{3em}  % prevent overfull lines
\setcounter{secnumdepth}{0}

\date{}

\begin{document}

\section{Codebook for Decapitated Animals Found in New York
Parks}\label{codebook-for-decapitated-animals-found-in-new-york-parks}

\subsection{Variable Descriptives}\label{variable-descriptives}

\subsubsection{animal}\label{animal}

variable: ANIMAL, variable type: character field (variable length),
variable label: the type of decapitated animal identified, note: in
cases where more than one type of animal was identified, only the
primary animal is listed.

\subsubsection{quantity}\label{quantity}

variable: QUANTITY, variable type: integer, variable label: the number
of decapitated animals identified, note: in cases where the number of
decapitated animals is ambiguous the smallest possible estimate is
given. Note that though this variable may refer to parts of animals, it
is given as an integer.

\subsubsection{body\_part\_found}\label{bodyux5fpartux5ffound}

variable: BODY\_PART\_FOUND, variable type: character field (variable
length), variable label: the part of the decapitated animal identified.

\subsubsection{date\_started}\label{dateux5fstarted}

variable: DATE\_STARTED, variable type: date, variable label: The date
on which a report regarding the decapitated animal was first filed with
the New York City Department of Parks and Recreation.

\subsubsection{date\_closed}\label{dateux5fclosed}

variable: DATE\_CLOSED, variable type: date, variable label: The date on
which the New York City Department of Parks and Recreation either
resolved the issue regarding the decapitated animal, note: in cases
where the New York City Department of Parks and Recreation did not
resolve the issues regarding the decapitated animal, the date on which
it did not resolve the issue regarding the decapitated animal is given.

\subsubsection{source}\label{source}

variable: SOURCE, variable type: categorical (``311-Call Center'',
``Department of Parks and Recreation''), variable label: The avenue by
which the decapitated animal was originally identified.

\subsubsection{division}\label{division}

variable: DIVISION, variable type: categorical (``Borough Maintenance
Operations Office - Queens'', ``Borough Maintenance Operations Office -
Manhattan'', ``Borough Maintenance Operations Office - Bronx'',
``Borough Maintenance Operations Office - Brooklyn'', ``Borough
Maintenance Operations Office - Staten Island'', ``Department of Parks
and Recreation''), variable level: The division of the New York City
Department of Parks and Recreation to which the decapitated animal was
reported.

\subsubsection{form}\label{form}

variable: FORM, variable type: categorical (``DPR General Intake''),
variable label: This is the form they use, note: There is only one form.

\subsubsection{status}\label{status}

variable: STATUS, variable type: categorical (``assigned'', ``closed''),
variable label: This variable refers to whether The Department of Parks
and Recreation is currently on the case.

\subsubsection{priority}\label{priority}

variable: PRIORITY, variable type: ordinal, variable description: the
level of priority, note: no known priority levels beyond normal exist.

\subsubsection{complaint\_type}\label{complaintux5ftype}

variable: COMPLAINT\_TYPE, variable type: categorical (``animal in a
park''), variable description: the niche occupied in the Department of
Parks and Recreation's genre-classification of complaints, notes: full
complaints classification schema is currently unknown.

\subsubsection{descriptor\_1}\label{descriptorux5f1}

variable: DESCRIPTOR\_1, variable type: character field (variable
length), variable description: a rich description of the substance of
the complaint as reported.

\subsubsection{descriptor\_2}\label{descriptorux5f2}

variable: DESCRIPTOR\_2, variable type: character field (variable
length), variable description: a second rich description of the
substance of the complaint as reported, note: words run dry.

\subsubsection{complaint\_details}\label{complaintux5fdetails}

variable: COMPLAINT\_DETAILS, variable type: character field (variable
length), variable description: the details of the complaint, as heard by
the Department of Parks and Recreation.

\subsubsection{location\_type}\label{locationux5ftype}

variable: LOCATION\_TYPE, variable type: categorical (``Park''),
variable description: park, notes: park, park. park.

\subsubsection{park\_or\_facility}\label{parkux5forux5ffacility}

variable: PARK\_OR\_FACILITY, variable type: character (variable
length), variable description: the specific park or facility in which
the decapitated animal was reported to be, notes: the park could
theoretically be a facility, fyi, but it is not; some cases not
specified (code: ``not specified'').

\subsubsection{property\_number}\label{propertyux5fnumber}

variable: property\_number, variable type: array, variable description:
some nonsense from the Parks Department, note: missing values of
nonsense are blank.

\subsubsection{park\_district}\label{parkux5fdistrict}

variable: PARK\_DISTRICT, variable type: integer, variable description:
the number of the park district in which the complaint was filed.

\subsubsection{additional\_location\_details}\label{additionalux5flocationux5fdetails}

variable: additional\_location\_details, variable type: character field
(variable length), variable description: specification of the location
of decapitated animal within park, notes: values represented in all caps
where originally rendered in purple crayon.

\subsubsection{council\_district\_number}\label{councilux5fdistrictux5fnumber}

variable: DISTRICT\_NUMBER, variable type: integer, variable
description: remaining number of months you have to live. jk. new york
city council district in which the decapitated animal was identified.

\subsubsection{site\_street\_address}\label{siteux5fstreetux5faddress}

variable: SITE\_STREET\_ADDRESS, variable type: address, variable
description: Street address of the New York City Department of Parks and
Recreation site at which the decapitated animal was found.

\subsubsection{site\_borough}\label{siteux5fborough}

variable: SITE\_BOROUGH, variable type: categorical (``manhattan'',
``bronx'', ``brooklyn'', ``queens'', ``staten island''), variable
description: The borough in which the site in which the identified
decapitated animal is located.

\subsubsection{site\_city\_zip}\label{siteux5fcityux5fzip}

variable: SITE\_CITY\_ZIP, variable type: array, variable description:
The city and the zip code of the site in which the identified
decapitated animal is located.

\subsubsection{lat}\label{lat}

variable: LAT, variable type: numerical, variable description: latitude
of decapitated animal.

\subsubsection{long}\label{long}

variable: LONG, variable type: numerical, variable description: latitude
of decapitated animal.

\subsubsection{complaint\_type\_confirmed}\label{complaintux5ftypeux5fconfirmed}

variable: COMPLAINT\_TYPE\_CONFIRMED, variable type: categorical
(``animal in a park''), variable description: please.

\subsubsection{descriptor\_confirmed:}\label{descriptorux5fconfirmed}

variable: DESCRIPTOR\_CONFIRMED, variable type: character (variable
length), variable description: i can't. note: all initial DPR
descriptions perfectly capture facts on the ground.

\subsubsection{resolution\_action\_updated:}\label{resolutionux5factionux5fupdated}

variable: RESOLUTION\_ACTION\_UPDATED, variable type: date, time. note:
observations with values beginning with ``4/1'' for this variable should
be interpreted with caution.

\subsubsection{resolution\_description}\label{resolutionux5fdescription}

variable: RESOLUTION\_DESCRIPTION, variable type: character, (variable
length), variable description: resolution of the decapitated animal.

\subsubsection{time\_to\_action}\label{timeux5ftoux5faction}

variable: TIME\_TO\_ACTION, variable type: categorical (``Closed: No
Further Updates'', ``Past Due''), variable description: nothing to do
with time to action.

\subsection{Appendices}\label{appendices}

Appendix 1: Street addresses were used to identify the latitude and
longitude of the park. In cases where street addresses were missing LAT
and LONG were inferred according to a process of manual interpolation as
follows: first, the park name alone was used; in cases where the park
name was unavailable, the zip code in combination with the verbal
description of the park were used in concert with visual inspection of a
map.

Appendix 2: Department of Parks and Recreation reports procured through
FOILD program by Chen were transcribed by a freelancer named Hazel.

\end{document}
